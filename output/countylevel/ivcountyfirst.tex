\begin{table}[!htbp]\centering
\def\sym#1{\ifmmode^{#1}\else\(^{#1}\)\fi}
\caption{First Stage Regression Results for County-Level Local Government Incentives and Education Stimulus}
\begin{tabular}{l*{1}{c}}
\hline\hline
                    &\multicolumn{1}{c}{(1)}   \\
                    &Education Stimulus Per Capita   \\
\hline
Special Education Children&     1366.88***\\
                    &    (419.50)   \\
Ln(Real Incentives) &       -1.17   \\
                    &      (1.93)   \\
Competitive County=0&        0.00   \\
                    &         (.)   \\
Competitive County=1&      -33.91***\\
                    &     (10.09)   \\
John McCain Vote Share&       -6.79***\\
                    &      (1.13)   \\
Unemployment Rate   &       32.34***\\
                    &      (4.54)   \\
Labor Force         &      -19.93   \\
                    &     (52.79)   \\
Representative Partisanship&       24.97   \\
                    &     (15.61)   \\
Ln(Patent Count)    &       -0.95   \\
                    &      (1.37)   \\
Ln(M\&A Count)      &        2.18** \\
                    &      (0.98)   \\
Foreign Greenfield Investment&        7.15   \\
                    &     (22.42)   \\
Ln(Domestic Greenfield Investment)&        0.31   \\
                    &      (1.61)   \\
\hline
Observations        &     3049.00   \\
State-Fixed Effects &         yes   \\
\hline\hline
\multicolumn{2}{p{\linewidth}}{\footnotesize The dependent variable is the amount of education stimulus scaled by a county's labor force. The independent variable is the number of special education students in a county divided by the county's school age population. Models include state fixed-effects. Unless otherwise stated, variables are from 2008. Robust standard errors are in parentheses and clustered by state. * p$<$0.10, ** p$<$0.05, *** p$<$0.010}\\
\end{tabular}
\end{table}
