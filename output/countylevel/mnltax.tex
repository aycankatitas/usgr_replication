\begin{table}[!htbp]\centering
\def\sym#1{\ifmmode^{#1}\else\(^{#1}\)\fi}
\caption{Multinomial Logit Regression of Greenfield FDI on Stimulus and Tax Incentives}
\begin{tabular}{l*{2}{c}}
\hline\hline
                    &         (1)   &               \\
                    &Old_Counties   &New_Counties   \\
\hline
Ln(Tax Incentives)  &       -0.13   &        0.06   \\
                    &      (0.17)   &      (0.04)   \\
Ln(Education Stimulus)&        0.27   &        0.69***\\
                    &      (0.81)   &      (0.12)   \\
John McCain Vote Share&        0.04   &        0.01   \\
                    &      (0.07)   &      (0.01)   \\
Unemployment Rate   &        1.41*  &        0.09** \\
                    &      (0.77)   &      (0.04)   \\
Labor Force         &        0.60   &        0.04   \\
                    &      (1.36)   &      (0.72)   \\
Representative Partisanship&       -1.94   &        0.19   \\
                    &      (1.23)   &      (0.15)   \\
Ln(Patent Count)    &        0.83** &        0.03   \\
                    &      (0.37)   &      (0.02)   \\
Ln(M\&A Count)      &        0.14   &        0.02   \\
                    &      (0.10)   &      (0.03)   \\
Foreign Greenfield Investment&       28.85***&       -0.64   \\
                    &      (9.43)   &      (0.51)   \\
Ln(Domestic Greenfield Investment)&        0.08   &       -0.01   \\
                    &      (0.09)   &      (0.02)   \\
\hline
Observations        &     3059.00   &               \\
State-Fixed Effects &         yes   &               \\
\hline\hline
\multicolumn{3}{p{\linewidth}}{\footnotesize Multinomial logit coefficients estimated via maximum likelihood. The outcome is a categorical variable that indicates  never, old and new county status. The baseline category is never counties. The independent variable is tax incentives. The model include state-fixed effects. Robust standard errors are in parentheses and clustered by state. * p$<$0.10, ** p$<$0.05, *** p$<$0.010}\\
\end{tabular}
\end{table}
